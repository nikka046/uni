\documentclass{article}

\usepackage{amsthm}
\usepackage{amsmath}
\usepackage{amssymb}
\usepackage[utf8]{inputenc}
\usepackage[table]{xcolor}
\usepackage{lineno,hyperref}
\usepackage{import}
\usepackage{xifthen}
\usepackage{pdfpages}
\usepackage{transparent}
\usepackage{float}

\theoremstyle{definition}
\newtheorem{df}{Definition}[section]
\newtheorem{ex}{Example}[section]
\newtheorem{thr}{Theorem}

\newcommand{\incfig}[1]{%
    \def\svgwidth{\columnwidth}
    \import{./figures/}{#1.pdf_tex}
}



\begin{document}

\title{COMP1215 - Sets}
\author{Dominik Tarnowski}
\date{October 2019}
\maketitle

\tableofcontents

\section{What is a set?}


\begin{df}{Set}
	An \textbf{unordered} collection of \textbf{objects} with \textbf{no duplicates}.
\end{df}

\subsection{Defining a set}
To Define a set, we use the following syntax:
\[ Let\ X = \{1,2,3\} \]
This defines a set called X with elements 0, 2 and 3.\\

\subsection{Properties of a set}
Note, that a set \textbf{doesn't specify the order}, so:
\[ X = \{ 3,1,2 \}\]
The set also \textbf{doesn't have duplicates}, so they can be simply ignored:
\[ Let\ Y = \{apple, apple, 1,2,2,3,3, pi, \{3.12\}\} \]
\[ \therefore Y = \{apple, 1,2,3,pi,\{3.12\}\]
Notice how I used different types of data types in the above example. This is because a set is a collection of \textbf{objects}, therefore it can include anything.

\subsection{Cardinality}
For finite sets, the cardinality of a set is simply its size:
\begin{ex}
	\[Let\ X = \{1,2,3,5\} \]
	\[|X| = 4\]
\end{ex}

\subsection{Membership}
Given a set:
\[ Let\ X = \{...\} \]
If an element x is included within X:
\[x \in X\]
If it is not included:
\[x \notin X\]

\begin{ex}
	\[ Let\ X = \{1,2,3,4\}\]
	\[ 1 \in X\]
	\[0 \notin X\]
\end{ex}

\subsection{Subsets}
\begin{df}{Subset}
	A set A is a subset of set B if all elements of A are included within B.
\end{df}

To explain, given a set:
\[Let\ B = \{...\}\]
If another set A is a subset of B:
\[A \subseteq B\]
This also means that $B$ is a superset of A, which is just another way of writing it:
\[B \supseteq A\]

\begin{ex}
\[Let\ X = \{1,2,3\}\]
\[\{1\} \subseteq X\]
\[\{1,3\} \subseteq X\]
\[\{1,3,4\} \not\subseteq X\]
\end{ex}

\subsection{Proper subset}
A proper subset works exactly the same as a subset, except for it \textbf{cannot contain all elements of the superset}. \\
\\
If set A is a proper subset of B:
\[A \subset B\]

\begin{ex}
	\[Let\ X = \{1,2,3\}\]
	\[\{1,2\} \subset X\]
	\[\{1,2,3\} \not\subset X\]
\end{ex}

\subsection{Existing sets}
\begin{table}[h!]
	\centering
	\begin{tabular}{|c|c|c|}
		\hline
		Symbol & Name & Description \\
		\hline
		$\mathbb{R}$ & Real & any real number \\
		$\mathbb{N}$ & Natural & integer $> 0$ \\
		$\mathbb{Z}$ & Integer & whole number\\
		2 & Binary Set & $2 = \{0,1\}$ \\
		\hline
	\end{tabular}
\end{table}
\section{Ordered Pairs}
Unline sets, ordered pairs can only contain two elements and their order matters. \\
\[(1,0) \neq (0,1)\]

\begin{ex}
	\[Let\ X = \{(0,1), (1,1)\}\]
	Note that ordered pairs can also contain duplicates.
\end{ex}

\section{Set Operations}
\begin{df}{Set Operation}
	Takes two sets and returns a single set.
\end{df}

\subsection{Union: $A \cup B$}
\begin{figure}[H]
    \centering
    \incfig{union}
\end{figure}

\subsection{Intersection: $A \cap B$}
\begin{figure}[H]
    \centering
    \incfig{intersection}
\end{figure}

\subsection{Difference: $A - B$}
\begin{figure}[H]
    \centering
    \incfig{difference}
\end{figure}




\subsection{Cartesian Product: $A \times B$}
Gives a 2D matrix representation of two sets, using a set of pairs.
\[A \times B = \{(a,b) \mid a \in A\ and\ b \in B\}\]

\begin{table}[h!]
	\centering
	\begin{tabular}{c | c c c c}
		$$ & $b_1$ & $b_2$ & ... & $b_m$ \\
		\hline
		$a_1$ & $(a_1, b_1)$ & $(a_1, b_2)$ & ... & $(a_1, b_m)$ \\
		$a_2$ & $(a_2, b_1)$ & $(a_2, b_2)$ & ... & $(a_2, b_m)$ \\
		... & ... & ... & ... & ... \\
		$a_n$ & $(a_n, b_1)$ & $(a_n, b_2)$ & ... & $(a_n, b_m)$ \\
	\end{tabular}
\end{table}

\begin{ex}
	\[Let\ A = \{a,b\}\]
	\[Let\ B = \{1,2,3\}\]
	\[\therefore A \times B = \{(a,1), (a, 2), (a,3), (b,1), (b,2), (b,3)\}\]
	
	\begin{table}[H]
		\centering
		\begin{tabular}{c | c c c }
		$$ & 1 & 2 & 3\\
		\hline
		a & (a,1) & (a,2) & (a,3) \\
		b & (b,1) & (b,2) & (b,3)
	\end{tabular}
	\end{table}
\end{ex}


\subsection{Sum / Disjoint Union: $A + B$}
Disjoint union takes two sets, A and B and unions them such that all elements from both sets are included:
\[X + Y = \{(x,0) \mid x \in X\} \cup \{(y,1) \mid y \in Y\}\]

\begin{ex}
	\[Let\ X = \{1,2,3\}\]
	\[Let\ Y = \{2,3,4\}\]
	Notice that the union skips the duplicates:\[X\cup Y = \{1,2,3,4\}\] However the disjoint union keeps them: \[X + Y = \{(1,0), (2,0), (3,0), (2,1), (3,1), (4,1)\}\]
\end{ex}

\section{Relations}
A relation from set X to set Y is \textbf{some set} of \textbf{pairs from their cartesian product}, so:
\[R \subseteq X \times Y\]

\subsection{Number of relations}
The number of relations is simply the number of subsets in the cartesian product. \\
The number of items in the cartesian product of sets X and Y is $|X| \times |Y|$. \\
So the number of relations must be:
\[2^{|X|\times |Y|}\]


\subsection{Identity Relation: $I_X$}
An identity relation is an example of a relation that maps a value of a set to the same value.\\
We use the notation $I_X$ where X is the set that the relation is on.
\begin{ex}
	\[Let\ X = \{1,2,3\}\]
	\[\therefore I_X: X \rightarrow X\ (more\ on\ this\ syntax\ later)\]
	\[\therefore I_X = \{(1,1), (2,2), (3,3)\}\]
\end{ex}

If you imagine the cartesian product as a 2D table matrix, then the identity relation is simply the diagonal: 
\begin{table}[h!]
\centering
\begin{tabular}{c | c c c c}
	$$ & $x_1$ & $x_2$ & ... & $x_m$ \\
	\hline
	$x_1$ & \cellcolor[HTML]{ffe2e0} $(x_1, x_1)$ & $(x_1, x_2)$ & ... & $(x_1, x_m)$ \\
	$x_2$ & $(x_2, x_1)$ & \cellcolor[HTML]{ffe2e0} $(x_2, x_2)$ & ... & $(x_2, x_m)$ \\
	... & ... & ... & \cellcolor[HTML]{ffe2e0} ... & ... \\
	$x_n$ & $(x_n, x_1)$ & $(x_n, x_2)$ & ... & \cellcolor[HTML]{ffe2e0}$(x_n, x_m)$ \\
\end{tabular}
\end{table}

\subsection{Equivalence relations}
An equivalence relation is a special relation that satisfies the following conditions:
\begin{enumerate}
	\item \textbf{reflexivity}: $\forall x \in X, (x,x) \in \sim $
	\item \textbf{symmetry}: $\forall x,y \in X, if\ (x,y)\ \in\ \sim\ then\ (y,x) \in\ \sim$
	\item \textbf{transitivity}: $x,y,z \in X,\ for\ (x,y) \in\ \sim\ and\ (y,z)\ then\ (x,z) \in\ \sim$
\end{enumerate}

\begin{ex}
	\[Let\ A = \{1,2,3\}\]
	Then a valid equivalence relation, $R$, can defined as follows: \\
	\[	\{ (1,1), \\
		(2,2),\\
		(3,3),\\
		(1,3),\\
		(3,1)\\
	\}
\]
	Of course, there are many other equivalence relations that can be defined, as long as they follow the given 3 rules.
\end{ex}

\subsubsection{Equivalence classes}
An equivalence class i a set of elements a given element is equivalent to. \\
Given a set X and element a such that $a \in X$:
\[ [a] = \{ x \mid x \in X\ and\ (x,a) \in\ \sim \}\]

\begin{ex}
	\[Let\ X = \{1,2,3\}\]
	\[Let\ \sim\ = \{ (1,1), (2,2), (3,3), (1,3), (3,1) \}\]
	\[\therefore [1] = \{1,3\}, [2] = \{2\}, [3] = \{3,1\}\]
\end{ex}

\subsubsection{Quotients}
\begin{df}{Quotient}
	The set of equivalence classes for each element in the given set.
\end{df}
A quotient of X with respect to ~ is usually denoted as:
\[X/\sim\ = \{[x] \mid x \in X\}\]

\subsection{Functions as relations}
A function (function graph to be specific, but we will cover that later) is simply a relation (called \textbf{f}) that satisfies the following conditions:
\begin{enumerate}
	\item $\forall x \in X$, there exists $y \in Y$ such that $(x,y) \in f$
	\item if $(x,y) \in f$ and $(x,z) \in f$ then $y=z$
\end{enumerate}
The following points can be easily simplied:
\begin{enumerate}
	\item function is defined on the entire domain
	\item function can only have 1 output for a given input
\end{enumerate}

\section{Functions}
A function maps from one set of values to a different set. The same x-value in a set cannot map to two different y-values, but two x-values can map to the same y-value.

\begin{figure}[H]
	\centering
	\incfig{func}
\end{figure}

\subsection{Range vs Codomain}
\begin{df}{Range.}
	All of Y values that are mapped to by X.
\end{df}
This can also be written as follows:
\[ \{f(x) \mid x \in X\} \]

\subsection{Declaring functions}
In discrete mathematics, a funciton can only have 1 input and 1 output.\\
\begin{ex}
Let's see an example of a function declaration.\\
A function called \textbf{f} that takes any integer, multiplies it by two and returns it would be defined as such:
\[f: \mathbb{Z} \rightarrow \mathbb{Z} \]
\[f(x) = x \times 2\]
We need to define what set the domain and codomain are before we write the equation for the function.\\
\end{ex}
\begin{ex}
A function that takes more than 1 input, for example a function \textbf{g} that multiplies two real numbers together would have to take a pair from the cartesian product of the two real sets:

\[g: \mathbb{R} \times \mathbb{R} \rightarrow \mathbb{R}\]
\[g(a,b) = a \times b\]

And if a function took more than two arguments, it would look somewhat like this:
\[h: (\mathbb{R} \times \mathbb{R}) \times \mathbb{R} \rightarrow \mathbb{R}\]
\[h(a,b,c) = a \times b + c\]
Note, that $h(a,b,c)$ is just a nicer way of writing $h((a,b), c)$ and this is actually what is happening under the hood.
\end{ex}

\subsection{Number of functions between two sets}

\subsection{Composing functions}
\begin{df}{Composition.}
	Taking output from one function and feeding it to another one
\end{df}
\[g(f(x)) = (f;g)(x) = (f \circ g)(x)\]
\subsection{Injective, surjective and bijective}
Functions can have their own properties. For example, a function can be injective, surjective, or both.

\subsubsection{Injective}
In an injective function, no 2 of the same X values give the same Y value.
\[\forall x,x' \in X,\ if\ f(x)=f(x')\ then\ x=x'\]

\subsubsection{Surjective}
In a surjective function, all of the codomain is being mapped to by the function.\\
This means that \textbf{range = codomain}.
\[let\ f: X \rightarrow Y\]
\[ \forall y \in Y,\ there\ exists\ x \in X\ such\ that\ f(x)=y\]

\subsubsection{Bijective}
A function is bijective if it's both injective and surjective.\\
This is also known as \textbf{1-1 correspondence}.

\subsection{Isomorphism theorem}
\begin{thr}
	All bijective functions have an inverse.
\end{thr}
\subsection{Graph of a function}
\begin{df}{function graph}
	Set of ordered pairs relating the inputs to the outputs of the function.
\end{df}

Given a graph, $F = \{(0,0), (1,0), (2,1), (3, 3)\}$, we can represent it with a diagraph:
\begin{figure}[H]
	\centering
	\incfig{diagraph}
\end{figure}
\subsection{Function spaces}
\begin{df}{function space.}
	A set of functions.
\end{df}

Given sets X and Y, the set of all functions from X to Y is defined by:
\[Y^X\]
This allows us to define functions like this one:
\[f: X \times Y^X \rightarrow Y\]
\[f(x,g) = g(x+4)\]

\section{Powersets}
\begin{df}{Powerset of X.}
	Set of all subsets of X.
\end{df}
Given a set X, we define the powerset with the following:
\[ P(X) \]
\begin{ex}
	\[P(\{0,1\}) = \{\emptyset, \{0\}, \{1\}, \{0, 1\}  \}\]
	The subset $\{1,0\}$ is the same as $\{0,1\}$, therefore it's not included.
\end{ex}
\begin{ex}
	\[P(\emptyset) = \{\emptyset\}\]
\end{ex}
\begin{ex}
	\[PP(\emptyset) = \{\emptyset, \{\emptyset\}\}\]
\end{ex}
\subsection{Cardinality of powersets}
The number of subsets of set X is given by:
\[2^{|X|}\]

\subsection{Partitions}
Given a set X and subsets $X_0, X_1, ..., X_n$, if:
\[X_0 \cup X_1 \cup ... \cup X_n = X\]
Then the given set of subsets is called a partition.

\begin{thr}
If ~ is an equivalence relation on X, then $X/\sim$ is a partition of X.
\end{thr}


\end{document}
