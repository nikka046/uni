\documentclass{article}

\usepackage{bookmark}
\usepackage{amsthm}
\usepackage{amsmath}
\usepackage{amssymb}
\usepackage[utf8]{inputenc}
\usepackage[table]{xcolor}
\usepackage{lineno,hyperref}
\usepackage{import}
\usepackage{xifthen}
\usepackage{pdfpages}
\usepackage{transparent}
\usepackage{float}
\usepackage{stmaryrd}
\usepackage{bussproofs}

\theoremstyle{definition}
\newtheorem{df}{Definition}[section]
\newtheorem{ex}{Example}[section]
\newtheorem{thr}{Theorem}
\newtheorem{pr}{Proposition}

\newcommand{\incfig}[1]{%
  \def\svgwidth{\columnwidth}
  \import{./figures/}{#1.pdf_tex}
}

\begin{document}
  \title{COMP1215 - Combinatorics}
  \author{Dominik Tarnowski}
  \date{11 November 2019}
  \maketitle

  \tableofcontents


  \section{Inclusion and Exclusion}
  \begin{enumerate}
    \item $|A \cup B| = |A| + |B| - |A \cap B|$
  \end{enumerate}

  \section{Pigeonhole principle}
  If $|A| > |B|$ (size of A $>$ size of B), then every function from $A$ to $B$ maps at least 2 distinct elements of $A$ to the same element of $B$. \\
  In other words:
  \[\text{For } f:A \to B\ \exists x_1,x_2 \in A, x_1 \neq x_2 \text{ such that } f(x_1) = f(x_2) \]
  
  \begin{ex}
    \[A = \{x \mid x \in \mathbb{Z}\ and\ |x| \leq 5\}\]
    \[B = \{x \mid x \in \mathbb{N}\ and\ x \leq 5\}\]
    As you can see, $|A|<|B|$, so all valid functions between $A$ and $B$ must map at least one pair of $x$ values to the same $y$ value.
    \[f: A \to B\]
    \[f(x) = x^2\]
    \[f(-5) = 25\]
    \[f(5) = 25\]
  \end{ex}



\end{document}

