\documentclass{article}

\usepackage{bookmark}
\usepackage{amsthm}
\usepackage{amsmath}
\usepackage{amssymb}
\usepackage[utf8]{inputenc}
\usepackage[table]{xcolor}
\usepackage{lineno,hyperref}
\usepackage{import}
\usepackage{xifthen}
\usepackage{pdfpages}
\usepackage{transparent}
\usepackage{float}
\usepackage{stmaryrd}
\usepackage{bussproofs}

\theoremstyle{definition}
\newtheorem{df}{Definition}[section]
\newtheorem{ex}{Example}[section]
\newtheorem{thr}{Theorem}
\newtheorem{pr}{Proposition}

\newcommand{\incfig}[1]{%
  \def\svgwidth{\columnwidth}
  \import{./figures/}{#1.pdf_tex}
}

\begin{document}
  \title{COMP1215 - Combinatorics}
  \author{Dominik Tarnowski}
  \date{11 November 2019}
  \maketitle

  \tableofcontents


  \section{Inclusion and Exclusion}
  \begin{enumerate}
    \item $|A \cup B| = |A| + |B| - |A \cap B|$
  \end{enumerate}

  \section{Pigeonhole principle}
  If $|A| > |B|$ (size of A $>$ size of B), then every function from $A$ to $B$ maps at least 2 distinct elements of $A$ to the same element of $B$. \\
  In other words:
  \[\text{For } f:A \to B\ \exists x_1,x_2 \in A, x_1 \neq x_2 \text{ such that } f(x_1) = f(x_2) \]
  
  \begin{ex}
    \[A = \{x \mid x \in \mathbb{Z}\ and\ |x| \leq 5\}\]
    \[B = \{x \mid x \in \mathbb{N}\ and\ x \leq 5\}\]
    As you can see, $|A|>|B|$, so all valid functions between $A$ and $B$ must map at least one pair of $x$ values to the same $y$ value.
    \[f: A \to B\]
    \[f(x) = x^2\]
    \[f(-5) = 25\]
    \[f(5) = 25\]
  \end{ex}

  \begin{ex}
    \textbf{A human has a maximum of approx 200,000 hairs on his head. Prove that at least 2 people in London have exactly the same number of hairs on their head.}
    \[Let\ A = \{x \mid x \in \mathbb{N}\ and\ x \leq 8.9 \times 10^6\}\]
    \[Let\ B = \{x \mid x \in \mathbb{N}\ and\ x \leq 200,000\}\]
    $|A|>|B|$ therefore a function $f : A \to B$ must map at least 2 values in $A$ to the same $B$ value.
  \end{ex}

  \section{Fibonacci series and recursion}
  \[|A| = \text{cardinality of set } A = \#\{a|a \in A\}\]
  \subsection{Recurrence relation}
  \[f_n = f_{n-1} + f_{n-2}\]
  \[
    \begin{pmatrix}
      f_n \\
      f_{n-1}
    \end{pmatrix} = \begin{pmatrix}
    1 & 1 \\
    1 & 0 \\
    \end{pmatrix} \begin{pmatrix}
    f_{n-1} \\
    f_{n-2}
    \end{pmatrix}
  \]
  \subsection{Inductive proofs and recursion}
  $\forall n : P(n)$
  \begin{enumerate}
    \item \textbf{Base Case:} First, prove $P(1)$ is true
    \item \textbf{Induction Step} \\
      Assume $P(0), P(1), ..., P(n)$ are all true (Induction hypothesis)\\
      Use the assumption above to prove $P(n+1)$\\
      If $P(n+1) = True$, proven
  \end{enumerate}

  \begin{ex}
    \textbf{Prove that the sum of all natural numbers adds up to $\frac{n(n+1)}{2}$ using a proof by induction.}\\
    So we need to prove $0+1+2+...+n=\frac{n(n+1)}{2}$. Base case proof:
    \[Let\ n = 1 \therefore 0+1 = \frac{1(2)}{2} = 1 \therefore \text{works for }n=1\]
    Now, we assume $n=k$ holds (Induction Hypothesis):
    \[1+2+...+k=\frac{k(k+1)}{2}\]
    And we need to show $n=k+1$ works:
    \[1+2+...+k+(k+1) = (1+2+...+k)+(k+1) = \frac{k(k+1)}{2}+(k+1)\]
    \[=\frac{k(k+1) + 2(k+1)}{2} = \frac{k^2+3k+1}{2} = \frac{(k+1)(k+2)}{2} \qed\]
  \end{ex}

  \subsubsection{Base Case}
  This case proves that the property holds for the first item. It doesn't have to begin with $n=0$, it often starts with $n=1$ but it can start with any number within the given set.
  \subsection{Fibonacci Q-Matrix}
  The $Q$ matrix is defined by:
  
  \[Q = Q^1 = \begin{bmatrix}
    1 & 1 \\
    1 & 0
  \end{bmatrix}
  \]

The $n^{th}$ value is given by:\[
Q^n = \begin{bmatrix}
  F_{n+1} & F_n \\
  F_n & F_{n-1} \\
\end{bmatrix}
  \]

\end{document}

